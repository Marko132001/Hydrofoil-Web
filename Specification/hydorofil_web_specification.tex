\documentclass[12pt,a4paper]{article}
\usepackage[utf8]{inputenc}
\usepackage[T1]{fontenc}
\usepackage{amsmath}
\usepackage{amssymb}
\usepackage{graphicx}
\usepackage[croatian]{babel}
\usepackage{hyperref}
\linespread{1.5}

\begin{document}

	\begin{center}
		{\Huge Adria Hydrofoil Web - Specifikacija}
	\end{center}
	
	\section{Korišteni alati}
	
	\par{
		\textbf{Frontend}
		\begin{itemize}
			\item NextJS
			\item Css + TailwindCss
		\end{itemize}
	}
	\par{
		\textbf{Backend}
		\begin{itemize}
			\item Strapi CMS
		\end{itemize}
	}
	
	\section{Trenutne funkcionalnosti stranice i ciljevi}
	
		\par{S obzirom da je glavni cilj projekta redizajn prijašnje uptz web stranice, adria hydrofoil stranica u trenutnom stanju sadrži sve slike i tekst koji se nalaze na uptz stranici. Neki dijelovi stranice su slični uptz stranici, te je stoga trenutni cilj korištenjem TailwindCss-a napraviti moderniji dizajn komponenti napravljenih u NextJS-u i Css-u. Za backend dio se koristi Strapi CMS preko kojeg se dohvaća većina sadržaja koji se prikazuje na stranici. Jedna funkcionalnost koja radi preko Strapija je implementacija bloga, odnosno zamišljeno je da admin stranice može preko Strapija jednostavno dodavati blog objave koje se automatski refreshaju na frontendu. Također na stranici je implementiran kontaktni obrazac koji korisnik može ispuniti, te pritiskom na gumb taj se obrazac automatizmom šalje na predodređeni mail. Uz novi dizajn komponenti stranice, potrebno je i promijeniti rezoluciju slika radi bržeg učitavanja.}
		\\
		\\
		{\large Linkovi:}
		\begin{itemize}
			\item \href{https://uptz.hr/}{UPTZ stranica}
			\item \href{https://hydrofoil-web-nextjs.vercel.app/}{Adria Hydrofoil stranica}
		\end{itemize}
		
	
\end{document}